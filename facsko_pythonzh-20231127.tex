\documentclass[a4paper,12pt]{letter}

\usepackage{amsmath}
\usepackage{amssymb}
\usepackage{nopageno}
\usepackage{amsmath}
\usepackage[a4paper, total={7.5in, 11.25in}]{geometry}

\begin{document}

%\pagestyle{empty}

Név: \hrulefill\ Neptun kód: \hrulefill

\begin{center}
 (VH-MIT0009-N) Python programozás alapok  2.~zárthelyi dolgozat
 \end{center}

 \begin{enumerate}
 \item Függvények, rekurzió
   \begin{enumerate}
   \item Számítsa ki az első 1000 természetes szám összegét rekurzív függvénnyel és írja ki a képernyőre! (A főprogram meghív egy külön deklarált rekurzív függvény az n=1000 paraméterrel.) \hfill (5~pont)
   \item A Fibonacci-számok egy ismert sorozat elemei. A sorozat nulladik eleme 0, az első eleme 1, a további elemeket az előző kettő összegeként kapjuk. Képlettel:
     \begin{equation*}
     F_{n}=\begin{cases}
    0, & \text{ha $n=0$}.\\
    1, & \text{ha $n=1$}.\\
    F_{n-1}+F_{n-2}, & \text{ha $n\ge1$.}
  \end{cases}
\end{equation*}
A Fibonacci-számok végtelen, növekvő sorozatot alkotnak; ennek első néhány eleme: 0, 1, 1, 2, 3, 5, 8, 13, 21, 34, 55, 89, 144\dots Számolja ki a sorozat első 100 elemét és írja ki a képernyőre. Deklaráljon egy olyan függvényt, amely kiszámítja a következő elemet az előző két elem alapján, majd egy ciklussal menjen végig az indexváltozón. \hfill (5~pont)
\item Számolja ki a fenti Fibonacci-sorozat első 100 elmét rekurzióval és írja ki őket a képernyőre. Deklaráljon egy olyan függvényt, amely saját magát hívja meg, amíg nem számítja ki az eredményt. \hfill (5~pont)
   \end{enumerate}
 \item Állományok beolvasása, ábrázolás, adatelemzés.
\begin{enumerate}
\item Az oktatójuk 2023.~szeptember 16-án szombaton délután kerékpározni ment. Az mellkasi pulzusmérőjéhez csatlakozott Polar Beat alkalmazás a \texttt{facsko\_bevprogzh-20231127.txt} szö\-ve\-ges állományban rögzítette a rekreációs edzés adatait. Az állomány első oszlopát és az időbélyegeket figyelmen kívül hagyva, olvassák be a földrajzi szélesség és hosszúság, illetve a magassági adatokat! \hfill (5~pont)
\item Ábrázolják az edzés útvonalát úgy, hogy átszámítják a földrajzi fokokat km-re. (Feleljen meg 1\textdegree\ a síkban 111\,km távolságnak. Euklideszi geometriában, azaz síkban dolgozzanak. Használják a Pitagorasz-tételt a valós elmozdulás kiszámítására.) Hány km-t tekert az oktatójuk? Az ábrának legyenek feliratozott tengelyei és címe is. A szöveges választ kommentként adják meg a forráskódban. \hfill (10~pont)
\item Mi volt az oktatójuk sebessége az edzés alatt? Készítsen egy idő-sebesség diagrammot! Az ábrának legyen címe és a tengelyeket is feliratozzák. Az időlépések 1\,s-nek felelnek meg. A sebességet az egyes pontok közötti elmozdulás és az 1\,s hányadosaként kapják. Ha km-ben számolták a távolságot, akkor km/s mértékegységben. Ha méterben, akkor m/s-ban. A m/s-os egységet 3.6-tal kell szorozni, hogy km/h-t kapjanak. \hfill (10~pont)
  \item Az idő mekkora részében száguldott az oktatójuk 0-5\,km/h, 5-10\,km/h, 10-15\,km/h, 15-20\,km/h, 20\,km/h vagy nagyobb sebességgel? Készítsen oszlopdiagrammot és olvassa le róla az eredményt. Az oszlopdiagrammnak legyen címe és feliratozza a tengelyeket (vagy oszlopokat). \hfill (10~pont)
 \end{enumerate}

 \end{enumerate}

A két zárhelyi dolgozatra összesen maximum 100 pont adható, 50-64 pont elégséges (2), 65-70 pont közepes (3), 71-84 pont jó (4) és 85-100 pont jeles (5). Ha a feladat nem teljesíthető a hallgatói gépterem munkaállomásainak konfigurációs hibái miatt, vagy azért, mert nem egyértelműen kitűzött, akkor az adott feladat mindenki számára maximális pontszámmal veendő figyelembe.

\bigskip

\rightline{Facskó Gábor}
\rightline{\textit{facsko.gabor@uni-milton.hu}}
\leftline{Budapest, 2023. november 27.}

\vfill

\end{document}
